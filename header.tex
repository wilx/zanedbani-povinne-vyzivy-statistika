\usepackage{amssymb}
\usepackage{newunicodechar}
\usepackage{accsupp}
\usepackage{MnSymbol}
\usepackage{textcomp}
\usepackage{etoolbox}
\usepackage{unicode-math}
\usepackage{cstypo}

\cstypoSingleLetterEnable
\cstypoALetterEnable
\cstypoParagraphEnable
\cstypoPercentsEnable

\AtBeginEnvironment{quote}{\slshape}

% This sets default image width.
\setkeys{Gin}{width=.85\textwidth}

%\setmainfont{Charis SIL}
\setmathfont[Scale=MatchLowercase]{TeX Gyre Schola Math}
\setmathfont[range={`A-`Z,`a-`z,`0-`9}]{Charis SIL}
%\setmainfont{TeX Gyre Pagella}
%\setmainfont{DejaVu Serif}
%\setsansfont{DejaVu Sans}
%\setmonofont{DejaVu Sans Mono}
%\newfontfamily\myurlfont[Ligatures={TeX}]{Liberation Sans Narrow}
\newfontfamily\myurlfont[Ligatures={TeX}]{DejaVu Serif Condensed}
\renewcommand\UrlFont{\myurlfont}

% This makes sure that linked text's links shows as a footnote.
\makeatletter
\let\oldhref=\href
\renewcommand\href[2]{\oldhref{#1}{#2}\footnote{\url{#1}}}
%\DeclareRobustCommand{\hreffootnote}{\hyper@normalise\hreffootnote@}
\makeatother

\newcommand{\noncopy}[1]{%
    \BeginAccSupp{method=escape,ActualText={}}%
    #1%
    \EndAccSupp{}%
}



% cleveref seems to break polyglossia for me
%\usepackage{cleveref}

% This allows line breaks in URL in more places.
\def\UrlBreaks{\do\/\do-\do.\do=\do_\do?\do\&\do\%\do\a\do\b\do\c\do\d\do\e\do\f\do\g\do\h\do\i\do\j\do\k\do\l\do\m\do\n\do\o\do\p\do\q\do\r\do\s\do\t\do\u\do\v\do\w\do\x\do\y\do\z\do\A\do\B\do\C\do\D\do\E\do\F\do\G\do\H\do\I\do\J\do\K\do\L\do\M\do\N\do\O\do\P\do\Q\do\R\do\S\do\T\do\U\do\V\do\W\do\X\do\Y\do\Z\do\0\do\1\do\2\do\3\do\4\do\5\do\6\do\7\do\8\do\9}

%\usepackage{lineno}
%\linenumbers

\usepackage[all]{nowidow}
\usepackage[above, section]{placeins}

\makeatletter
\def\fps@figure{p}
\makeatother
